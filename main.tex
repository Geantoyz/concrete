%!TEX TS-program = xelatex

\documentclass[a4paper,14pt]{article}

\usepackage{header}


\author{}
\title{}
\date{\today}

\begin{document} % конец преамбулы, начало документа
\section{Компонування поперечної рами будівлі}
\subsection{Компонування поперечної рами промислової будівлі}
Визначаємо висоту підкранової балки: при кроці 6 м:

$h_{\textit{п,б}}=1000$ мм

Визначити висоту надкранової $H_{\textit{в}}$ і підкранової $H_{\textit{н}}$ частин колони, повну висоту $H_{\textit{1}}$, $H$.


\newpage
\section{Статичний розрахунок поперечної рами}
\begin{equation}
    A_f\geqslant\frac{1,05N_{n,max}}{R_0-\gamma_mH_1} 
\end{equation}

$$A_f\geqslant\frac{1,05\cdot25}{200-2\cdot6}=0,14\;{\textit{м}^2}$$

\begin{equation}
    A_f\geqslant\frac{1,05N_{n,max}}{R_0-\gamma_mH_1} 
\end{equation}
\newpage
\section{Проектування колони одноповерхової промислової будівлі}
\subsection{Розрахунок поздовжньої арматури колони}


\subsection{Розрахунок розпірки двогілкової колони}

\subsection{Розрахунок колони із площини поперечної рами}
\newpage
\section{Проектування позацентрового навантаження фундаменту під колону}
\subsection{Визначення розмірів фундаменту і армування його плитної частини}
\subsection{Проектування підколонника фундаменту}

\newpage
\section{Проектування плити покриття}
\subsection{Розрахунок міцності поздовжніх ребер плити покриття за нормальними перерізами}
\subsection{Розрахунок міцності похилих перерізів поздовжніх ребер плити}
\subsection{Розрахунок полички плити на місцевий вигин}
\subsection{Розрахунок втрат попереднього напруження}
\subsection{Розрахунок плити на утворення тріщин нормальних до поздовжньої осі}
\subsection{Розрахунок тріщиностійкості плити в стадії виготовлення і транспортування}
\subsection{Розрахунок плити за деформаціями}

\newpage
\section{Проектування кроквяної ферми}



\end{document} % конец документа