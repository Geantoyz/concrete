%!TEX TS-program = xelatex

\documentclass[a4paper,14pt]{article}

\usepackage{header}


\author{}
\title{}
\date{\today}

\begin{document} % конец преамбулы, начало документа
\section{Компонування поперечної рами будівлі}
\subsection{Компонування поперечної рами промислової будівлі}
\begin{enumerate}
\item Визначаємо висоту підкранової балки: при кроці 6 м:

$h_{\textit{п,б}}=1000$ мм

\item Визначити висоту над кранової $H_{\textit{в}}$ і підкранової $H_{\textit{н}}$ частин колони, повну висоту $H_{\textit{1}}$, $H$.

Вантажопідйомність $Q= 20\;{\textit{т}}$.

Висота $A=2400\;{\textit{мм}}$.

$H_{\textit{н}}=8600\;{\textit{мм}}$.

h кранового рельса =$70\;{\textit{мм}}$.

$H_{\textit{в}}=h_{\textit{п.б.}}+A+1000$=$1000+2400+100=3500\;\textit{мм}$.

$H_1=H_{\textit{н}}+H_{\textit{в}}=8000+3500=12100\;\textit{мм}$.

$H=H_1+150=12250\;\textit{мм}$.

Висота ферми при прольоті 18\;\textit{м}\;:

$H_{\textit{ф}}=2450\;\textit{мм}$.

\item Прив'язка "а" розбивочной осі ряду колон:

-- нульова прив'язка.

\item Призначити висоту перетину над кранової частини колони $h_{\textit{верхнє}}$:

При нульовій прив'язці --- $380\;{\textit{мм}}$.

$h_{\textit{нижнє}}=(\frac{1}{10}\ldots\frac{1}{14})H_\textit{н}=860\ldots614\;\textit{мм}$.

$b_{\textit{нижнє}},\;b_{\textit{верхнє}}=(\frac{1}{20}\ldots\frac{1}{25})H_\textit{н}=430\ldots344\;\textit{мм}$.

Вид колони --- наскрізна.

Так як :$H_1<\textit{10,8}\;\textit{м}$; $h_\textit{нижнє}<\textit{900}\;\textit{м}$; $Q<\textit{30}\;\textit{т}$, проліт до 24 м, то приймаємо розміри колони:

$h_{\textit{гілки}}=200\;\textit{мм}$.

$h_{\textit{н}}=1000\;\textit{мм}$.

$b_{\textit{нижнє}},\;b_{\textit{верхнє}}=400\;\textit{мм}$.
%Сделать рисунок
\end{enumerate}
\newpage
\section{Статичний розрахунок поперечної рами}




Збір навантаження:

%Сделать таблицу

%Рис. 3.1.  Статична розрахункова схема рами промислових будівель:
%Рис. 3.2. Розрахункова схема кроквяної конструкції при визначенні опорної реакції RA
Розрахунковий проліт рами:

$l_0=L_{\textit{цеха}}-2с=17000-2\cdot 200=16600\;{\textit{мм}}$

Визначення опорної реакції $R^{\textit{Пост}}_A$:

\begin{equation}
    R^{\textit{Пост}}_A=0,5\cdot g^{\textit{покр}}\cdot l_0 + {\textit{1,1}}\cdot {\textit{0,5}}\cdot G^{\textit{стр}}_{\textit{П}}
\end{equation}

де : $G^{\textit{стр}}_{\textit{П}}$ - маса кроквяної конструкції

$g^{\textit{покр}}$ - навантаження на покритті

 
\begin{equation}
    g^{\textit{покр}}=g_{\textit{р}}\cdot S_1
\end{equation}

де : $g_{\textit{р}}$ - розрахункове постійне навантаження на 1 м$м^{\textit{2}}$ плити покриття

$S_1$-крок поперечних рам в будівлі

$g^{\textit{покр}}={\textit{3,52}}\cdot 6 = {\textit{21,12}}\;\textit{кН/м}$

$R^{\textit{Пост}}_A=0,5\cdot {\textit{21,12}}\cdot {\textit{16,6}}+ {\textit{1,1}}\cdot {\textit{0,5}}\cdot 60={\textit{208,296}}\;{\textit{кН}}$

Снігове навантаження

\begin{equation}
    p^{\textit{сн}}=S_{\textit{m}}\cdot S
\end{equation}
\begin{equation}
    S_m=\gamma_{fm}\cdot S_0 \cdot C
\end{equation}

де : $\gamma_{fm}$- коеф. надійності для середн. періоду повтрюваності снігового навантаження Т = 60 років 

$S_0$ - характеристичне значення снігового навантаження на 1 м$м^{\textit{2}}$ для заданого району будівництва

C = 1 при відсутності даних про режим експлуатації будівлі с плоскою конструкцією покрівлі і розміщенням його на висоті H < ${\textit{0,5}}$ км над рівнем моря.

$S_m=1,04\cdot 1400\cdot 1=1456\;\textit{Па}=1,456\;{\textit{кН/м}}^2$

$p^{\textit{сн}}=1,456\cdot 6=8,736\;\textit{кН/м}$

\begin{equation}
    R^{\textit{cн}}_A=0,5\cdot p^{\textit{сн}}\cdot l_0
\end{equation}

$R^{\textit{cн}}_A=0,5\cdot 8,736 \cdot 16,6 = 72,51\;\textit{кН/м}$

Кранове навантаження

Проліт крана $L_k$=$\textit{16,6}\;\textit{м}$

Ширина крана $B=6300\;\textit{мм}$

База крана $K=4400\;\textit{мм}$

$H=2400\;\textit{мм}$

$B_1=260\;\textit{мм}$

$P^n_{max}$-навантаження коліс на підкранові рейки-$195\;\textit{кН}$

Вага візка - $8,5\;\textit{т}$

G - Вага крана з візком -$28,5\;\textit{т}$

Тип кранової рейки - КР70

%вставить рисунок линий влияния


\begin{equation}
    D_{max}=\gamma_{fm}\cdot \psi \cdot P^n_{max} \cdot \sum y_i
\end{equation}

де: $\gamma_{fm}$ - см. п. 7.9 %[]

$\psi$ - см. п. 7.22 %[]

$\sum y_i$ - Рис..

\begin{equation}
    D_{min}=\gamma_{fm}\cdot \psi \cdot P^n_{min} \cdot \sum y_i
\end{equation}

\begin{equation}
P^n_{min} = \frac {Q+G}{n_0}-P^n_{max}
\end{equation}

де : $n_o$ - кількість коліс на одній стороні крана

$D_{max}=1,1\cdot 0,85 \cdot 195 \cdot 1,95$ = $355,534\;\textit{кН}$

$P^n_{min} = \frac {200+285}{2}-195$=$47,5\;\textit{кН}$

$D_{min}=1,1\cdot 0,85 \cdot 47,5 \cdot 1,95$ = $86,6\;\textit{кН}$

Навантаження на раму від поперечного гальмування

\begin{equation}
    T=\gamma_\textit{соч}\cdot \gamma_f \cdot T^\textit{кол}_{n} \cdot \sum y_i
\end{equation}

Горизонтальне поперечне гальмівне навантаження від одного колеса 
   для кранів з гнучким підвісом вантажу 

\begin{equation}
    T^\textit{кол}_{n}=\frac {0,05\cdot (Q+Q_t)}{n_0}
\end{equation}

$T^\textit{кол}_{n}=\frac {0,05\cdot (20+8,5)}{2}=\textit{0,7125}\;\textit{т}=\textit{7,2}\;\textit{кН}$

$T=0,85\cdot 1,2 \cdot 7,2 \cdot 1,95 = \textit{14,32}\;\textit{кН}$

Навантаження від стінових панелей:

\begin{equation}
    G_\textit{стпн}= S \cdot Н_\textit{н}\cdot g
\end{equation}

$G_\textit{стпн}= 6 \cdot 8,6\cdot 2,8 = 144,48\;\textit{кНм}$

\begin{equation}
    G_\textit{стпн.в.}= S \cdot Н_\textit{в}\cdot g
\end{equation}

$G_\textit{стпн.в.}= 6 \cdot 3,5\cdot 2,8= 58,8\;\textit{кНм}$

Вітрове навантаження

Граничне розрахункове значення вітрового навантаження:

\begin{equation}
    W_\textit{m}= \gamma_{fm} \cdot W_\textit{0}\cdot C
\end{equation}

де:  $\gamma_{fm}$ ---  коефіцієнт надійності, в залежності від терміну повторності максимального значення вітрового тиску в роках. На 100 років --- $\gamma_{fm}$ = 1,14

$W_\textit{0}$ --- характеристичне значення вітрового тиску, залежне від району будівництва. $W_\textit{0} - 0,47\;\textit{кНм}^2$

$h=5\;\textit{м} = W_{5}= 0,47\cdot 0,4 = 0,188\;\textit{кНм}^2$

$h=10\;\textit{м} = W_{10}= 0,47\cdot 0,6 = 0,282\;\textit{кНм}^2$

$h=20\;\textit{м} = W_{20}= 0,47\cdot 0,85 = 0,399\;\textit{кНм}^2$

Еквівалентне вітрове навантаження $W_e$

\begin{equation}
    W_\textit{e}= \frac{2M_3}{H^2}
\end{equation}

$M_3 = \frac{0,188\cdot 12,25^2}{2}+\frac {1}{2}\cdot (0,308-0,188)\cdot 7,25 \cdot (\frac {2}{3}\cdot 7,25 + 5)= 18,4\;\textit{кНм}^2$ 

$W_\textit{e}= \frac{2\cdot 18,4}{12,25^2}= 0,245\;\textit{кНм}^2$ 

Активний вітер 

\begin{equation}
    W_\textit{a}= W_\textit{e} \cdot B\cdot C_{aer}\cdot \gamma_{fm} 
\end{equation}

$W_\textit{a}= 0,245 \cdot 6\cdot 0,8\cdot 1,14 = 1,341\;\textit{кН/м.п.}$   

Пасивний вітер

$W_\textit{п}= 0,245 \cdot 6\cdot 0,6\cdot 1,14 = 1,01\;\textit{кН/м.п.}$ 

Зосереджена сила на рівні верха колон по середньому вітряному тиску між $0,308\;\textit{кНм}^2$ і $0,337\;\textit{кНм}^2$

$W = (\frac{0,308+0,337}{2})\cdot 6 \cdot 2,45 \cdot (0,8+0,6)\cdot 1,14 = 7,57\;\textit{кН}$ 

%Вставить рисунок разреза ветра

Статична розрахунок поперечної рами 

%<<<<<<< HEAD
\begin{enumerate}
    \item Момент інерції відносно осі Y:
    \begin{equation}
        I_z = \frac{b\cdot h^3_\textit{в}}{12}+\frac{bh-(H_\textit{н}-h_\textit{в})^2}{2} 
    \end{equation}
    
    $$I_z = \frac{40\cdot 20^3}{12}+\frac{40\cdot 20-(100-20)^2}{2}=23866,66\;\textit{см}^4$$

    $EF=3310000\cdot (0,4\cdot 0,2) = 264800\;\textit{т}$

    $I_y = 40 \cdot 20\cdot 40^2 = 0,0064\;\textit{см}^2$

    $EI_y = 3310000\cdot 0,064 = 21184$
%Вставить рисунок с сечением колон

    \item Розрахункове поєднання зусиль
    
    \textbf{Елемент 1, переріз 1}
    \begin{equation*}
        \begin{aligned}
        1+2&+3+4-7\qquad  &1+&3+6+7 &1+2&+3+4-7\\
        M^+_y &= \textbf{+265,99} &M^-_y &= \textbf{-193,405}\qquad &N^-_{max} &= -719,658\\
        N_\textit{відп} &= -719,658  &N_\textit{відп} &= -597,47 &M_\textit{відп} &= +265,99 \\
        Q_\textit{z.відп} &= -41,052 &Q_\textit{z.відп} &= 18,078 &Q_\textit{z.відп} &= -41,052 \\
        \dfrac{M}{N}&=0,369  &\dfrac{M}{N}&=0,323 &\dfrac{M}{N}&=0,369 
        \end{aligned}
    \end{equation*}
 
    \textbf{Елемент 1, переріз 2}
    \begin{equation*}
        \begin{aligned}
        1+2&+3+6-8\qquad  &1+&2+3+4\\
        M^-_y &= -103,237 &N^-_{max} &= -682,55 \\
        N_\textit{відп} &= -629,63  &M_\textit{відп} &= -39,418\\
        Q_\textit{z.відп} &= 3,701 &Q_\textit{z.відп} &= 22,14 \\
        \dfrac{M}{N}&=0,16  &\dfrac{M}{N}&=0,05  
        \end{aligned}
    \end{equation*}
        \textbf{Елемент 3, переріз 1}
        \begin{equation*}
        \begin{aligned}
        1&+2+4  &1&+6 &1&+2\\
        M^+_y &= \textbf{+72,771} &M^- &= -3,62 &N^-_{max} &= -316,008\\
        N_\textit{відп} &= -308,311\qquad  &N_\textit{відп} &= -239044\qquad &M_\textit{відп} &= +39,742 \\
        Q_\textit{z.відп} &= -22,14 &Q_\textit{z.відп} &= 2,835 &Q_\textit{zвідп} &= -10,888 \\
        \dfrac{M}{N}&=0,23  &\dfrac{M}{N}&=0,01 &\dfrac{M}{N}&=0,12
        \end{aligned}
        \end{equation*}
        \textbf{Елемент 3, переріз 2}
        \begin{equation*}
        \begin{aligned}
        &1+2  &1&+2+4\\
        N^-_{max} &= -301,046\qquad &Q_z &= -10,888 \\
        Q^-_\textit{z.відп} &= 17,735 &N^-_\textit{відп} &= -293,349 \\
        \end{aligned}
        \end{equation*}

        Від постійного навантаження

        \textbf{Елемент 1, переріз 1}
        \begin{equation*}
            \begin{aligned}
            &\quad1  \\
            N &= 277,49 \\
            M_y &= 26,653  \\
            Q &= -8,242 \\
            \end{aligned}
            \end{equation*}
        \textbf{Елемент 1, переріз 2}
        \begin{equation*}
            \begin{aligned}
            &\quad1  \\
            N &= -240,381 \\
            M_y &= -44,228  \\
            Q &= -8,242 \\
            \end{aligned}
            \end{equation*}
        \textbf{Елемент 3, переріз 1}
        \begin{equation*}
            \begin{aligned}
            &\quad1  \\
            N &= -239,044 \\
            M_y &= 30,083  \\
            Q &= -8,242 \\
            \end{aligned}
            \end{equation*}
            \textbf{Елемент 3, переріз 2}
        \begin{equation*}
            \begin{aligned}
            &\quad1  \\
            N &= -277,049 \\
            M_y &= -26,65+3  \\
            Q &= -8,242 \\
            \end{aligned}
            \end{equation*}
\end{enumerate}
%=======

%>>>>>>> c66bf3aeb27969b09cea77f323300f71c1fbe802


\newpage
\section{Проектування колони одноповерхової промислової будівлі}
\subsection{Розрахунок поздовжньої арматури колони}

\begin{enumerate}
    \item Обчислюємо ексцентриситет:
        \begin{equation}\label{eq:e_0}
            e_0 = \frac{M}{N}+e_a
        \end{equation}
        де: \begin{itemize}
                \item $e_a = \frac{1}{600} \cdot 8600 = 14,3\;\textit{мм}$
                \item $e_a = \frac{1}{30} \cdot 200 = 6,6\;\textit{мм}$
            \end{itemize}
        Обираємо $e_a = 14,3\;\textit{мм}$
        $$e_0 = \frac{265,99}{219,688}+0,014 = 0,384\;\textit{м}$$
    \item Наведений радіус інерції перерізу підкранової частини двогілкової колони:
        \begin{equation}
            i_{red}^2 = \frac{c^2}{4(\frac{1+3c^2}{\psi^2n^2h^2})}
        \end{equation}
        де: $\psi^2 = 1,5$

                $n = \frac{H_\textit{н}}{S} = \frac{8,6}{2} = 4,3\;\textit{м}$

                $S = (8 \ldots 10)h = 10 \cdot 0,2 = 2\;\textit{м}$
           
        $$i_{red}^2 = \frac{0,8^2}{4(\frac{1+3 \cdot 0,8^2}{1,5 \cdot 4,3^2 \cdot 0,2^2})} = 0,05859\;\textit{м}$$
    \item Приведена гнучкість підкранової частини колони:
        \begin{equation}
            \lambda_{red} = \frac{l_0}{i_{red}^2}
        \end{equation}
        де: $l_0 = 1,5H_\textit{н} = 1,5 \cdot 8,6 = 12,9\;\textit{м}$
        $$\lambda_{red} = \frac{12,9}{0,05859} = 220,17$$
        
        Гранична гнучкість:
        \begin{equation}
            \lambda \lim = \frac{20ABC}{\sqrt{n}}
        \end{equation}
        де: $n = \frac{N}{A_cf_{cd}} = \frac{719,658 \cdot 10^3}{2(0,4 \cdot 0,2) \cdot 17 \cdot 10^6} = 0,265$

            $A = \frac{1}{(1+0,2\phi_{ef})} = \frac{1}{(1+0,2 \cdot 2)} = 0,71$

            $\phi_{ef} = 2$

            $B = 1,1$

            $C = 0,7$
            $$\lambda\lim = \frac{20 \cdot 0,71 \cdot 1,1 \cdot 0,7}{\sqrt{0,265}} = 21,61$$

        Так як, $\lambda_{red} > \lambda\lim$ слід враховувати вплив прогину на величину ексцентриситету повздовжньої сили. В цьому випадку в розрахунку замість $e_0$ необхідно використовувати
        величину $(\eta \cdot l_0)$, де 
        \begin{equation}
            \eta = \frac{1}{1 - \frac{N}{N_{cr}}}
        \end{equation}
        \begin{equation}
            N_{cr} = \frac{6,4E_{cm}}{l_0^2}\left[\frac{I}{\phi_l}\left(\frac{0,11}{0,1 + \dfrac{\sigma_e}{\phi_p}} + 0,1\right) + \alpha I_s\right]
        \end{equation}
        $I = 2\left[\frac{bh^3}{12} + bh(\frac{c}{2})^2\right] = 2\left[\frac{0,4 \cdot 0,2^3}{12} + 0,4 \cdot 0,2 (\frac{0,8}{2})^2\right] = 0,02613\;\textit{м}^4$

        $\phi_l = 1 + \beta \frac{M_1}{M} = 1 + 1 \cdot \frac{26,653}{265,99} = 1,1 < (1 + \beta)$

        $I_S = 2 \rho bh(\frac{c}{2})^2 = 2 \cdot 0,02 \cdot 0,4 \cdot 0,2 \cdot (\frac{0,8}{2})^2 = 0,000512\;\textit{м}^4$

        $\sigma_e = \frac{l_0}{h_\textit{н}} = \frac{12,9}{1} = 12,9\;\textit{м}$

        $\phi_p = 1$

        $\alpha = \frac{E_S}{E_{ct}} = \frac{210\;\textit{Па}}{32,5\;\textit{Па}} = 6,46$
        $$N_{cr} = \frac{6,4 \cdot 32500 \cdot 10^6}{12,9^2}[\frac{0,02613}{1,1}(\frac{0,11}{0,1 + \frac{12,9}{1}} + 0,1) + 6,46 \cdot 0,000512]$$
        $$N_{cr} = 7354530\;\textit{Па} = 7354,53\;\textit{кН}/\textit{м}^2$$
        $$\eta = \frac{1}{1 - \frac{719,658}{7354,53}} = 1,11$$
    \item Визначаємо зусилля в гілках колони:
        \begin{equation}
            N_{\textit{в}1,2} = 0,5N \pm \frac{M \cdot \eta}{c}
        \end{equation}
        $$N_{\textit{в}1,2} = 0,5\cdot 719,658 + \frac{265,99 \cdot 1,1}{0,8} = 713,2\;\textit{кН}$$
        \begin{equation}
            M_\textit{в} = V \frac{S}{4}
        \end{equation}
        $$M_\textit{в} = 41,052 \cdot \frac{2}{4} = 20,526\;\textit{кН}$$
    \item Для кожної з гілок визначаємо:
        \begin{equation}
            e_0 = \frac{M_\textit{в}}{N_\textit{в}} + l_a
        \end{equation}
        \begin{equation}
            e = e_0\eta +  0,5h - a
        \end{equation}
        $\eta = 1$

        $h = 200\;\textit{мм}$

        $a = 30\;\textit{мм}$

        $d = h - a = 200 - 30 = 170\;\textit{мм}$

        $l_a = 200 / 30 = 6,6\;\textit{мм}$

        $\frac{S}{600} = \frac{2000}{600} = 3,33\;\textit{мм}$
        $$e_0 = \frac{20,526}{713,2} + 0,0066 = 0,035\;\textit{м}$$
        $$e = 0,035 \cdot 1 +  0,5 \cdot 0,2 - 0,03 = 0,105\;\textit{м}$$
    \item Підбираємо армування при несиметричному армуванні:
        \begin{equation}
            A_S^\prime = \frac{N \cdot e - 0,4 \cdot f_{cd} \cdot b \cdot d^2}{f_{yd} \cdot (d - a^\prime)} \geqslant 0
        \end{equation}
        $$A_S^\prime = \frac{713,2 \cdot 10^3 \cdot 0,105 - 0,4 \cdot 17 \cdot 10^6 \cdot 0,4 \cdot 0,17^2}{365 \cdot 10^6 \cdot (0,17 - 0,03)} \geqslant 0$$
        $$A_S^\prime = -0,0000728376\;\textit{м}^2$$

        Висновок --- переріз арматури приймаємо конструктивно.
        \begin{equation}
            A_S = \frac{0,55 \cdot f_{cd} \cdot b \cdot d - N}{f_{yd}} + A_S^\prime
        \end{equation}
        $$A_S = \frac{0,55 \cdot 17 \cdot 10^6 \cdot 0,4 \cdot 0,17 - 713,2 \cdot 10^3}{365 \cdot 10^6} +(-0,0000728376)$$
        $$A_S = -0,000284892\;\textit{м}^2$$

        Висновок --- переріз арматури приймаємо конструктивно.

        Підбираємо арматуру за відсотком армування:

        $A_S^\prime$ приймаємо $4\varnothing12A400C$ --- $A = 4,52\;\textit{см}^2$

        $A_S$ приймаємо $4\varnothing12A400C$ --- $A = 4,52\;\textit{см}^2$
        \begin{equation}
            \rho = \frac{A_S^\prime + A_S}{b \cdot d} \cdot 100\%
        \end{equation}
        $$\rho = \frac{4,52 + 4,52}{40 \cdot 20} \cdot 100\% = 1,13\%$$
        Оптимальне значення армування для колон 1\ldots 3\%    
\end{enumerate}
%Добавить ссылки на формулы
\textbf{Розрахунок за другою комбінацією зусиль.}
\begin{enumerate}
    \item Обчислюємо ексцентриситет за формулою (\ref{eq:e_0}):
        
        де: \begin{itemize}
                \item $e_a = \frac{1}{600} \cdot 8600 = 14,3\;\textit{мм}$
                \item $e_a = \frac{1}{30} \cdot 200 = 6,6\;\textit{мм}$
            \end{itemize}
        Обираємо $e_a = 14,3\;\textit{мм}$
        $$e_0 = \frac{193,405}{597,47}+0,014 = 0,34\;\textit{м}$$
    \item Наведений радіус інерції перерізу підкранової частини двогілкової колони (link):  
        $$i_{red}^2 = \frac{0,8^2}{4(\frac{1+3 \cdot 0,8^2}{1,5 \cdot 4,3^2 \cdot 0,2^2})} = 0,05859\;\textit{м}$$
    \item Приведена гнучкість підкранової частини колони (link):
        $$\lambda_{red} = \frac{12,9}{0,05859} = 220,17$$
        
        Гранична гнучкість (link):

        де: $n = \frac{N}{A_cf_{cd}} = \frac{597,47 \cdot 10^3}{2(0,4 \cdot 0,2) \cdot 17 \cdot 10^6} = 0,22$

            $A = \frac{1}{(1+0,2\phi_{ef})} = \frac{1}{(1+0,2 \cdot 2)} = 0,71$

            $\phi_{ef} = 2$

            $B = 1,1$

            $C = 0,7$
            $$\lambda\lim = \frac{20 \cdot 0,71 \cdot 1,1 \cdot 0,7}{\sqrt{0,22}} = 23,31$$

        Так як, $\lambda_{red} > \lambda\lim$ слід враховувати вплив прогину на величину ексцентриситету повздовжньої сили. В цьому випадку в розрахунку замість $e_0$ необхідно використовувати
        величину $(\eta \cdot l_0)$ за формулами (link) (link)
        
        $I = 0,02613\;\textit{м}^4$

        $\phi_l = 1 + 1 \cdot \dfrac{44,228}{193,405} = 1,23 < (1 + \beta)$

        $I_S = 0,000512\;\textit{м}^4$

        $\sigma_e = 12,9\;\textit{м}$

        $\phi_p = 1$

        $\alpha = 6,46$
        $$N_{cr} = \frac{6,4 \cdot 32500 \cdot 10^6}{12,9^2}\left[\frac{0,02613}{1,23}\left(\frac{0,11}{0,1 + \dfrac{12,9}{1}} + 0,1\right) + 6,46 \cdot 0,000512\right]$$
        $$N_{cr} = 7014160\;\textit{Па} = 7014,16\;\textit{кН}/\textit{м}^2$$
        $$\eta = \dfrac{1}{1 - \dfrac{597,47}{7014,16}} = 1,1$$
    \item Визначаємо зусилля в гілках колони за формулами (link), (link):
        $$N_{\textit{в}1,2} = 0,5 \cdot 597,47 - \frac{193,405 \cdot 1,1}{0,8} = 32,803\;\textit{кН}$$
        $$M_\textit{в} = 18,078 \cdot \frac{2}{4} = 9,039\;\textit{кН}$$
    \item Для кожної з гілок за формулами (link), (link) визначаємо:

        $\eta = 1$

        $h = 200\;\textit{мм}$

        $a = 30\;\textit{мм}$

        $d = 170\;\textit{мм}$

        $l_a = 6,6\;\textit{мм}$

        $\dfrac{S}{600} = 3,33\;\textit{мм}$
        $$e_0 = \frac{9,039}{32,803} + 0,0066 = 0,28\;\textit{м}$$
        $$e = 0,28 \cdot 1 +  0,5 \cdot 0,2 - 0,03 = 0,35\;\textit{м}$$
    \item Підбираємо армування при несиметричному армуванні за формулами\\(link),(link):
        $$A_S^\prime = \frac{32,803 \cdot 10^3 \cdot 0,35 - 0,4 \cdot 17 \cdot 10^6 \cdot 0,4 \cdot 0,17^2}{365 \cdot 10^6 \cdot (0,17 - 0,03)} \geqslant 0$$
        $$A_S^\prime = -0,00131364\;\textit{м}^2$$

        Висновок --- переріз арматури приймаємо конструктивно.
        $$A_S = \frac{0,55 \cdot 17 \cdot 10^6 \cdot 0,4 \cdot 0,17 - 32,803 \cdot 10^3}{365 \cdot 10^6} +(-0,00131364)$$
        $$A_S = 0,000338407,\;\textit{м}^2$$

        Висновок --- переріз арматури приймаємо конструктивно.

        Підбираємо арматуру за відсотком армування (link):

        $A_S^\prime$ приймаємо $4\varnothing12A400C$ --- $A = 4,52\;\textit{см}^2$

        $A_S$ приймаємо $4\varnothing12A400C$ --- $A = 4,52\;\textit{см}^2$
        $$\rho = \frac{4,52 + 4,52}{40 \cdot 20} \cdot 100\% = 1,13\%$$
        Оптимальне значення армування для колон 1\ldots 3\%
\end{enumerate}
\textbf{Розрахунок надкранової частини колони} %Сделать эскиз
\begin{enumerate}
    \item Обчислюємо ексцентриситет за формулою (\ref{eq:e_0}),
    де: \begin{itemize}
            \item $e_a = \frac{1}{600} \cdot H_\textit{в} =\frac{1}{600} \cdot 3500 = 5,83\;\textit{мм}$
            \item $e_a = \frac{1}{30} \cdot 380 = 12,6\;\textit{мм}$
        \end{itemize}
    Обираємо $e_a = 12,6\;\textit{мм}$
    $$e_0 = \frac{72,771}{308,311}+0,0126 = 0,25\;\textit{м}$$
\item Наведений радіус інерції перерізу підкранової частини двогілкової колони:
    \begin{equation}
        i_{red} = 0,289 \cdot h
    \end{equation}
    $$ i_{red} = 0,289 \cdot 0,38 = 0,11\;\textit{м}$$
\item Приведена гнучкість підкранової частини колони за формулою (link):
    
    де: $l_0 = 2H_\textit{в} = 2 \cdot 3,5 = 7\;\textit{м}$
    $$\lambda_{red} = \frac{7}{0,11} = 63,63$$
    Гранична гнучкість за формулою (link):
    
    де: $n = \dfrac{N}{A_cf_{cd}} = \dfrac{308,311 \cdot 10^3}{0,4 \cdot 0,38 \cdot 17 \cdot 10^6} = 0,12$

        $A = \dfrac{1}{(1+0,2\phi_{ef})} = \dfrac{1}{(1+0,2 \cdot 2)} = 0,71$

        $\phi_{ef} = 2$

        $B = 1,1$

        $C = 0,7$
        $$\lambda\lim = \frac{20 \cdot 0,71 \cdot 1,1 \cdot 0,7}{\sqrt{0,12}} = 31,56$$

    Так як, $\lambda_{red} > \lambda\lim$ слід враховувати вплив прогину на величину ексцентриситету повздовжньої сили. В цьому випадку в розрахунку замість $e_0$ необхідно використовувати
    величину $(\eta \cdot l_0)$ за формулами (link) (link), де: 
    
    $I = \dfrac{b \cdot h^3}{12} = \dfrac{0,4 \cdot 0,38^3}{12} = 0,0126\;\textit{м}^4$

    $\phi_l = 1 + \beta \dfrac{M_1}{M} = 1 + 1 \cdot \dfrac{30,083}{72,771} = 1,41 < (1 + \beta)$

    $I_S = \rho \cdot \left(\dfrac{d - a}{h}\right)^2 = 0,02 \cdot \left(\dfrac{0,35 - 0,03}{0,38}\right)^2 = 0,0142\;\textit{м}^4$

    $\sigma_e = \dfrac{l_0}{h_\textit{н}} = \dfrac{7}{0,38} = 18,42\;\textit{м}$

    $\sigma_{min} = 0,5 - 0,01 \cdot \dfrac{\sigma_e}{h} - 0,01f_{cd} =  0,5 - 0,01 \cdot \dfrac{18,42}{0,38} - 0,01 \cdot 17 = -0,155$

    $\phi_p = 1$

    $\alpha = \dfrac{E_S}{E_{ct}} = \dfrac{210\;\textit{Па}}{32,5\;\textit{Па}} = 6,46$
    $$N_{cr} = \frac{6,4 \cdot 32500 \cdot 10^6}{7^2}\left[\frac{0,0126}{1,41}\left(\frac{0,11}{0,1 + \frac{18,42}{1}} + 0,1\right) + 6,46 \cdot 0,0142\right]$$
    $$N_{cr} = 393412000\;\textit{Па} = 393412\;\textit{кН}/\textit{м}^2$$
    $$\eta = \frac{1}{1 - \frac{308,311}{393412}} = 1$$
  
\item Підбираємо армування при симетричному армуванні:
    \begin{equation}
        A_S = A_S^\prime = \frac{N \cdot e_0 - f_{cd} \cdot b \cdot h \cdot (d - 0,5h)}{f_{yd} \cdot (d - a^\prime)} \geqslant 0
    \end{equation}
    $$A_S = A_S^\prime = \frac{308,311 \cdot 10^3 \cdot 0,25 - 17 \cdot 10^6 \cdot 0,4 \cdot 0,38 \cdot (0,35 - 0,5 \cdot 0,38)}{365 \cdot 10^6 \cdot (0,35 - 0,03)}$$
    $$A_S = A_S^\prime = -0,0028\;\textit{м}^2$$

    Висновок --- переріз арматури приймаємо конструктивно.

    $A_S = A_S^\prime$ приймаємо $4\varnothing12A400C$ --- $A = 4,52\;\textit{см}^2$
\end{enumerate}
\subsection{Розрахунок розпірки двогілкової колони}
    \begin{enumerate}
        \item Згинальний момент в розпірці
        \begin{equation}
            M_{ds} = \pm \frac{V \cdot s}{2} 
        \end{equation}
        $$M_{ds} = - \frac{41,052 \cdot 2}{2} = -41,052\;\textit{кНм}$$
        \item Необхідна площа поздовжньої арматури при симетричному армуванні без врахування роботи бетону
        \begin{equation}
            A_S = A_S^\prime = \dfrac{M_{ds}}{f_{yd} \cdot (d - a^\prime)}
        \end{equation} 
        $$A_S = A_S^\prime = - \dfrac{41,052 \cdot 10^3}{365 \cdot 10^6 \cdot (0,36 - 0,04)} = -0,0003514\;\textit{м}^2$$
        $A_S = A_S^\prime$ приймаємо $3\varnothing14A400C$ --- $A = 4,61\;\textit{см}^2$
        \item Поперечна сила в розпірці
        \begin{equation}
            V_{ds} = \dfrac{2M_{ds}}{c} = \dfrac{V \cdot s}{c}
        \end{equation}
        $$V_{ds} = \dfrac{41,052 \cdot 2}{0,8} = 102,63\;\textit{кН}$$
        \item Умова необхідності розрахунку поперечних стрижнів розпірки
        \begin{equation}
            V_{Rd,c} = \left[C_{Rd,c} \cdot k \cdot \left(100 \cdot \rho_1 \cdot f_{ck,prism}\right)^{1/3}\right] \cdot b \cdot d
        \end{equation}
        де: $k = 1 + \sqrt{\dfrac{200}{14}} = 4,78$, приймаємо $2$

        $f_{ck,prism} = 22$

        $C_{Rd,c} = 0,12$

        $\rho_1 = \dfrac{A_S}{b \cdot d} = \dfrac{4,61}{40 \cdot 36} = 0,003$
        $$V_{Rd,c} = \left[0,12 \cdot 2 \cdot \left(100 \cdot 0,003 \cdot 22 \cdot 10^6 \right)^{1/3}\right] \cdot 0,4 \cdot 0,36 = 6,48\;\textit{кН}$$
        $$V_{ds}\nleq V_{Rd,c}$$
        Умова не виконується.
        \begin{equation}
            V_{Rd,max} = \dfrac{\alpha_{cw} \cdot b_w \cdot z \cdot v \cdot f_{cd}}{\cot\theta + \tan\theta}
        \end{equation}
        де: $\alpha_{cw} = 1$

        $z = 0,9 \cdot d = 0,9 \cdot 0,36 = 0,324$

        $v = 0,6 \cdot \left(\dfrac{1 - f_{ck,prism}}{250}\right) \leq 0,6$

        $v = 0,6 \cdot \left(\dfrac{1 - 22}{250}\right) = 0,54 \leq 0,6$

        $\cot\theta = 2,5$

        $\tan\theta = 0,4$
        $$ V_{Rd,max} = \dfrac{1 \cdot 0,4 \cdot 0,324 \cdot 0,54 \cdot 17 \cdot 10^6}{2,5 + 0,4} = 410251\;\textit{кН}$$
        $$V_{Rd,max} > V_{Rd,c}$$
        Приймаємо крок поперечної арматури

        $S \leq 0,5h = 200\;\textit{мм}$

        $S \leq 150\;\textit{мм}$
        
        $S \leq S_{w,max} = 0,75d = 270\;\textit{мм}$
        
        Приймаємо $\varnothing6A240C$ з кроком $150$ мм.
    \end{enumerate}
\subsection{Розрахунок колони із площини поперечної рами}
Виявляємо необхідність розрахунку підкранової частини колони із площини поперечної рами 
\begin{equation}
    \lambda = \dfrac{l_0}{i}
\end{equation}
де: $l_0 = 0,8H_\textit{н} = 0,8 \cdot 8,6 = 6,88\;\textit{м}$

$i=\sqrt{\dfrac{b^2}{12}} = \sqrt{\dfrac{0,4^2}{12}} = 0,11$
$$\lambda = \dfrac{6,88}{0,11} = 62,54$$
Так як $\lambda_{red}>\lambda=220,17>62,54$ тому розрахунок не потрібен.
\newpage
\section{Проектування позацентрового навантаження фундаменту під колону}

На фундамент передаються зусилля, що виникають в нижньому перетині колони $M_{IV}$, $N_{IV}$, $V_{IV}$. При цьому враховувати три невигідно розрахункових поєднання. Розрахунок тіла фундаменту виконують на дію відпору (реактивного тиску) грунту $Р_{гр}$, що виникає під підошвою фундаменту.


Розрахунок фундаменту полягає у визначенні:
\begin{enumerate}

\item Розмірів підошви фундаменту $l \cdot b$;

\item Загальної висоти фундаменту $Н_і$ висоти нижньої ступені $h_1$;

\item Необхідної площі арматури сітки С-1, що укладається у підошви фундаменту;

\item Необхідної площі поздовжньої і поперечної арматури підколонника.
\end{enumerate}
    Для фундаментів приймати важкий бетон класів С12/15…С20/25; робочу арматуру сітки С-1 классів А400, A300 ($\varnothing10-\varnothing18 \textit{мм}$) з кроком 100÷250 мм.

\subsection{Визначення розмірів фундаменту і армування його плитної частини}
%сделать ескиз 
\begin{enumerate}
    \item Призначаємо величину $H_1$ з умов: 
    \begin{itemize}
        \item $H_1 \geqslant H_{an} + 200 + 150 + 50$
        \item $H_1 \geqslant h_f$
    \end{itemize}
    $H_an$  для колон з двогілкової підкрановою частиною:
    \begin{itemize}
        \item $H_{an} \geqslant 0,33h_n + 500 = 0,33 \cdot 1000 + 500 = 830\;\textit{мм}$
        \item $H_{an} \geqslant 1,5h = 1,5 \cdot 200 = 300\;\textit{мм}$
        \item $H_{an} \geqslant 30d = 30 \cdot 12 = 360\;\textit{мм}$
    \end{itemize}
    Приймаємо $1100\;\textit{мм}$

     $H_1 \geqslant 1100 + 200 + 150 + 50 = 1500\;\textit{мм}$

     $H_1 \geqslant 1200\;\textit{мм}$

     Приймаємо $H_1 = 1950\;\text{мм}$.
    \item Попередньо приймаємо розміри фундаменту. %Сделать єскиз
    \item Визначаємо зусилля що діють на підставу фундаменту для трьох невигідних комбінацій зусиль в опорному перерізу колони
    \begin{equation}
        M = M_{IV} + V_{IV} \cdot (H_1 - 0,15) + G_{\textit{ст}} \cdot e_{\textit{ст}}
    \end{equation}
    \begin{equation}
        N = N_{IV} + G_{\textit{ст}}
    \end{equation}
    \textbf{Елемент 1 переріз 1}
    \begin{itemize}
        \item $1 + 2 + 3 + 4 - 7$
     
        $M_y^+ = 265,99\;\textit{кНм}$
        
        $N_{\textit{відп}} = - 719,658\;\textit{кН}$
   
        $Q_{\textit{відп}} = - 41,052\;\textit{кН}$
       \begin{equation}
           G_{\textit{ст}} = G_{\textit{фб}} \cdot \gamma_{sm}
       \end{equation}
       де $G_{\textit{фб}} = 1,8$ т.
   
       $\gamma_{sm} = 1,2$
       $$G_{\textit{ст}} = 18 \cdot 1,2 = 21,6\;\textit{кН}$$
       $e_{\textit{ст}} = \dfrac{t_{\textit{ст}} + h_c}{2} = \dfrac{300 + 1550}{2} = 925\;\textit{мм}$
       $$M_1 = 265,99 + 41,052 \cdot (1,95 - 0,15) + 21,6 \cdot 0,925 = 359,9\;\textit{кНм}$$
       $$N_1 = 719,658 + 21,6 = 741,26\;\textit{кН}$$
       \textit{Від нормативних значень:}
   
       $M_{n1} = \dfrac{265,99}{1,15} = 231,3\;\textit{кНм}$
   
       $N_{n1} = \dfrac{719,658}{1,15} = 625,8\;\textit{кН}$
       $$M_{N1} = 231,3 + 41,052 \cdot (1,95 - 0,15) + 21,6 \cdot 0,925 = 325,2\;\textit{кНм}$$
       $$N_{N1} = 625,8 + 21,6 = 647,4\;\textit{кН}$$
       \item $1 + 3 + 6 + 7$

        $M_y^- = - 193,405\;\textit{кНм}$
        
        $N_{\textit{відп}} = - 597,47\;\textit{кН}$
   
        $Q_{\textit{відп}} = 18,078\;\textit{кН}$
       $$M_2 = 193,405 + 14,078 \cdot (1,95 - 0,15) + 21,6 \cdot 0,925 = 245,93\;\textit{кНм}$$
       $$N_2 = 597,47 + 21,6 = 619,07\;\textit{кН}$$
       \textit{Від нормативних значень:}
   
       $M_{n2} = \dfrac{193,405}{1,15} = 168,18\;\textit{кНм}$
   
       $N_{n2} = \dfrac{597,47}{1,15} = 520\;\textit{кН}$
       $$M_{N2} = 168,18 + 18,078 \cdot (1,95 - 0,15) + 21,6 \cdot 0,925 = 220,7\;\textit{кНм}$$
       $$N_{N2} = 520 + 21,6 = 541,6\;\textit{кН}$$
    \end{itemize}
     Для подальших розрахунків використовуємо сполучення $1 + 2 + 3 + 4 - 7$.
    \item Визначаємо попередні розміри підошви фундаменту.
    \begin{equation}
        A_f \geqslant \dfrac{1,05 \cdot N_{n,max}}{R_0 - \gamma_m \cdot H_1}
    \end{equation}
     $m = \dfrac{b}{l} = 0,8 = \dfrac{2,7}{3,3}$

     $A_f = 3,3 \cdot 2,7 = 8,91\;\textit{м}^2$
     $$A_f \geqslant \dfrac{1,05 \cdot 647,4}{0,2 \cdot 10^3 - 20 \cdot 1,95} = 4,22\;\textit{м}^2$$
    \item Уточнюємо розрахунковий опір основи:
    \begin{equation}
        R = R_0 \cdot \left(1 + k_1 \cdot \dfrac{b-b_0}{b_0}\right) + k_2 \cdot \gamma \cdot \left(d - d_0\right)
    \end{equation}
    де $d = H_1 = 1,95\;\textit{м}$;

    $d_0 = 2\;\textit{м}$;

    $k_1$, $k_2$ для глинистих --- $0,05$, $0,15$ відповідно.

    Так як $d < d_0$ в вираженні для R другий додаток приймати рівним 0.
    $$R = 0,2 \cdot \left(1 + 0,05 \cdot \dfrac{2,4-1}{1}\right) = 0,215\;\textit{мПа}$$
    Різниця не суттєва. Перевіряти не потрібно.
    \item Для прийнятих розмірів підошви фундаменту обчислюємо геометричні характеристики:
    
    $A_f = 8,91\;\textit{м}^2$

    $W_f = \dfrac{bl^2}{6} = \dfrac{2,7 \cdot 3,3^2}{6} = 4,9$
    \item Для кожної з розрахункових комбінацій зусиль обчислюємо крайові напруги в ґрунті під підошвою фундаменту:
    
    \textit{Від нормативної:}
    \begin{equation}
        P_{n,max} = \gamma_m \cdot H_1 + \dfrac{N_{n,max}}{A_f} + \dfrac{M_{n,max}}{W_f}
    \end{equation}
    \begin{equation}
        P_{n,min} = \gamma_m \cdot H_1 + \dfrac{N_{n,max}}{A_f} - \dfrac{M_{n,max}}{W_f}
    \end{equation}
    \begin{equation}
        P_{n,mid} = \gamma_m \cdot H_1 + \dfrac{N_{n,max}}{A_f}
    \end{equation}
    \textbf{Для першого сполучення:}
    $$P_{n,max} = 20 \cdot 1,95 + \dfrac{647,4}{8,91} + \dfrac{325,6}{4,9} = 178,03\;\textit{кН}$$
    $$P_{n,min} = 20 \cdot 1,95 + \dfrac{647,4}{8,91} - \dfrac{325,6}{4,9} = 45,3\;\textit{кН}$$
    $$P_{n,mid} = 20 \cdot 1,95 + \dfrac{647,4}{8,91} = 111,7\;\textit{кН}$$
    \textbf{Для другого сполучення:}
    $$P_{n,max} = 20 \cdot 1,95 + \dfrac{541,6}{8,91} + \dfrac{220,7}{4,9} = 144,8\;\textit{кН}$$
    $$P_{n,min} = 20 \cdot 1,95 + \dfrac{541,6}{8,91} - \dfrac{220,7}{4,9} = 54,74\;\textit{кН}$$
    $$P_{n,mid} = 20 \cdot 1,95 + \dfrac{541,6}{8,91} = 99,8\;\textit{кН}$$
    \item Перевіряємо попередньо прийняті розміри підошви фундаменту з умов:
        \begin{equation}
        \begin{aligned}
            P_{n,max} &\leq 1,2R\\
            P_{n,min} &> 0\\
            P_{n,mid} &\leq R
        \end{aligned}
        \end{equation} 
    \textbf{Для першого сполучення:}
    \begin{equation*}
        \begin{aligned}
            178,03 &\leq 240\\
            45,3 &> 0\\
            111,7 &\leq 200
        \end{aligned}
        \end{equation*}
    \textbf{Для другого сполучення:}
    \begin{equation*}
        \begin{aligned}
            144,8 &\leq 240\\
            54,7 &> 0\\
            99,8 &\leq 200
        \end{aligned}
        \end{equation*}
    Остаточно приймаємо розміри фундаменту $b \times l = 2,7 \times 3,3\;\textit{м}$    
    \item Визначаємо напруження в ґрунті від розрахункових зусиль $M$ і $N$ без урахування мас ґрунту і фундаменту:
        \begin{equation}
        \begin{aligned}
            P_{max} &= \dfrac{N}{A_f} + \dfrac{M}{W_f}\\
            P_{min} &= \dfrac{N}{A_f} - \dfrac{M}{W_f}           
        \end{aligned}
        \end{equation}
        \textbf{Для першого сполучення:}
    \begin{equation*}
        \begin{aligned}
            P_{max} &= \dfrac{647,4}{8,91} + \dfrac{325,2}{4,9} = 139,03\;\textit{кН}\\
            P_{min} &= \dfrac{647,4}{8,91} - \dfrac{325,2}{4,9} = 6,3\;\textit{кН} 
        \end{aligned}
        \end{equation*}
    \textbf{Для другого сполучення:}
    \begin{equation*}
        \begin{aligned}
            P_{max} &= \dfrac{541,6}{8,91} + \dfrac{220,7}{4,9} = 105,8\;\textit{кН}\\
            P_{min} &= \dfrac{541,6}{8,91} - \dfrac{220,7}{4,9} = 15,74\;\textit{кН} 
        \end{aligned}
        \end{equation*}
    \item Перевіряємо достатність висоти $d_1$ нижньої сходинки з умов міцності по 
    поперечній силі в перерізі $2-2$ з урахуванням роботи тільки бетону (тобто без поперечного армування):
    \begin{equation}
        d_1 \geq \dfrac{P_{max} \cdot C}{f_{ctk,0,05}}
    \end{equation}
    \begin{equation}
        c=0,5 \cdot (l-a_n-2d)
    \end{equation}
    $$d = 450 - 50 = 400\;\textit{мм}$$
    $$c=0,5 \cdot (3,3 - 1,55 - 2 \cdot 0,4) = 0,475$$
    $$ d_1\geq \dfrac{139,04 \cdot 10^3 \cdot 0,475}{1 \cdot 10^6} = 0,066\;\textit{м}$$
    $$0,45 > 0,066$$
\end{enumerate} 
\subsection{Проектування підколонника фундаменту}
        $1 + 2 + 3 + 4 - 7$
     
        $M_y^+ = 265,99\;\textit{кНм}$
        
        $N_{\textit{відп}} = - 719,658\;\textit{кН}$
   
        $Q_{\textit{відп}} = - 41,052\;\textit{кН}$
\begin{enumerate}
    \item Зусилля в перерізі 7–7 підколонника:
        \begin{equation}
            M = M_{IV} + V_{IV}H_{an} + G_{cm}e_{cm}
        \end{equation}
        $$M = 265,99 + 41,052 \cdot 1,1 + 21,6 \cdot 0,925 = 331,13\;\textit{кНм}$$
        \begin{equation}
            N = N_{IV} + G_{cm} + G_1
        \end{equation}
        $$N = 719,658 + 21,6 + 24,15 = 765,408\;\textit{кН}$$
        де $G_1 = (0,84 \cdot 1,15) \cdot 2500 = 2415\;\textit{кг}$
    \item Необхідна площа поздовжньої арматури підколонника при\\$e_0 = M / N < 0,3h_{on} = 331,13 / 765,408 = 0,433 < 0,3 \cdot 1,51 = 0,453$:
        \begin{equation}
            A_S = A_S^{\prime} = \dfrac{Ne - f_{cd}S_0}{f_{yd}Z_S}
        \end{equation}
        де $e = e_0 + 0,5a_n - a = 0,433 + 0,5 \cdot 1,55 - 0,04 = 1,168$;

        $Z_S = h_n - 2a = 1550 - 2 \cdot 40 = 1470\;\textit{мм}$;

        $S_0 = 0,5 \cdot (b_nh_{on}^2 - bh_{\textit{н}}Z_S) = 0,5 \cdot (0,95 \cdot 1,51^2 - 0,4 \cdot 1) = 0,79\;\textit{м}$;

        $a = 30 \div 40\;\textit{мм}$.
        $$A_S = A_S^{\prime} = \dfrac{765,408 \cdot 1,168 - 14,5 \cdot 10^3 \cdot 0,79}{365 \cdot 10^3 \cdot 1,47} = - 0,01968\;\textit{м}^2$$
    \item Остаточно прийнятий поздовжня арматура підколонника повинна бути не менше конструктивного мінімуму:
        \begin{equation}
            A_S = A_S^{\prime} \geq \mu_{min} \cdot A_b = 0,001 \cdot (h_n \cdot b_n - h_{\textit{н}}b)
        \end{equation}
        $$A_S = A_S^{\prime} \geq 0,001 \cdot (1,55 \cdot 0,95 - 1 \cdot 0,4) = 0,0010725\;\textit{м}^2$$
        Приймаємо $6\varnothing16A400$ $A = 12,06\;\textit{см}^2$ з кроком $175\;\textit{мм}$.
    \item Необхідну площу поперечної арматури підколонника визначити з розрахунку міцності похилого перерізу на дію моменту за формулою залежно від $e_0$:
    
    При $\dfrac{h_{\textit{н}}}{6} < e_0 < \dfrac{h_{\textit{н}}}{2} = 0,16 < 0,453 < 0,5$
        \begin{equation}
            A_{sw} = \dfrac{M + VH_{an} - 0,7Ne_0 + G_{cm}(e_{cm} - 0,7e_0)}{f_{ywd} \sum Z_w}
        \end{equation}
        де $\sum Z_w = Z_1 + Z_2 + Z_3 + \ldots + Z_n = 50 + 200 + 350 + 650 + 800 + 950 +\\+ 1100 + 1250 = 5350\;\textit{мм}$
        \begin{multline*}
        A_{sw} = (265,99 \cdot 10^3 + 41,052 \cdot 10^3 \cdot 1,1 - 0,7 \cdot 719,658 \cdot 10^3 \times\\\times 0,453 + 21,6 \cdot (0,925 - 0,7 \cdot 0,453))/285 \cdot 10^6 \times\\\times 5,35 = 0,0000544\;\textit{м}^2
        \end{multline*}
        Приймаємо $4\varnothing12A400$ $A = 4,52\;\textit{см}^2$ з кроком $150\;\textit{мм}$.
\end{enumerate} %Вставить езкіз

\newpage
\section{Проектування плити покриття}
\subsection{Розрахунок міцності поздовжніх ребер плити покриття за нормальними перерізами}
Клас напруженої арматури $A800$

$f_{pk} = 840\;\textit{МПа}$

$f_{p0,1k} = 765\;\textit{МПа}$

$E_p = 190000\;\textit{МПа}$

$f_{pd} = \dfrac{f_{p0,1k}}{\gamma_s} = \dfrac{765}{1,2} = 637,5\;\textit{МПа}$

Монтажна арматура $A400$

$f_{yd} = 365\;\textit{МПа}$

Клас бетону $25/30$

$f_{cd} = 17\;\textit{МПа}$

$f_{ck,prizm} = 22\;\textit{МПа}$

$\epsilon_{cu3,cd} = 3\;\textit{МПа}$

$E_{cm} = 32,5 \cdot 10^3\;\textit{МПа}$
\begin{enumerate}
    \item Визначити відношення $h_f^\prime/h$, по ньому встановити величину   (ширину полички тавра за рис. 6.1 при наявності поперечних ребер), що вводиться в розрахунок.
        \begin{equation}
            b_{eff} = min\begin{Bmatrix}
                \dfrac{1}{6} \cdot l_k \\
                \textit{При}\;h_f^\prime \geq 0,1 \cdot h : 0,5 \cdot B_K - b
            \end{Bmatrix}
        \end{equation}
        $\dfrac{h_f^\prime}{h} = \dfrac{30}{300} = 0,1$

        $l_{\textit{к}} = L - 20 = 18000 - 20 = 17980\;\textit{мм}$

        При $h_f^\prime \geq 0,1 \cdot h$
        $$b_{eff} = 0,5 \cdot B_K - b = 0,5 \cdot 2,98 - 0,18 = 1,31\;\textit{м}$$
        Де $B_K = 2,98\;\textit{м}$;

        $b = 0,18\;\textit{м}$.
        $$l_0 = l_k - 2 \cdot \dfrac{2}{3} \cdot c = 17980 - 2 \cdot \dfrac{2}{3} \cdot 120 = 17820\;\textit{мм}$$
        $$M_{max} = \dfrac{q \cdot l_0^2}{8} = \dfrac{10,5291 \cdot 17,82^2}{8} = 417,9\;\textit{кНм}$$
        $$V_{max} = \dfrac{q \cdot l_0}{2} = \dfrac{10,5291 \cdot 17,82}{2} = 93,8\;\textit{кН}$$
        Де $q_p = q_1^{\textit{покр}} \cdot B = 2,0537 \cdot 3 = 6,1611\;\textit{кН/м}$;

        $P_{cm} = S_m \cdot B = 1,456 \cdot 3 = 4,368\;\textit{кН/м}$;

        $q = q_p + P_{cm} = 6,1611 + 4,368 = 10,5291\;\textit{кН/м}$.
    \item Обчислюємо $\alpha_m$:
        \begin{equation}
            \alpha_m = \dfrac{M_{max}}{b_{eff} \cdot d^2 \cdot f_{cd}}
        \end{equation}
        Де $d = h - a$;

        $a = 30 \div 50\;\textit{мм}$.
        $$\alpha_m = \dfrac{M_{max}}{b_{eff} \cdot d^2f_{cd}} = \dfrac{417,9 \cdot 10^3}{1,31 \cdot 0,26^2 \cdot 17 \cdot 10^6} = 0,277 \Rightarrow 0,274$$
        $\xi = \dfrac{x}{d} \Rightarrow x = \xi \cdot d = 0,41 \cdot 260 = 106,6\;\textit{мм}$

        $\zeta = 0,836$
    \item Попереднє напруження $\sigma_{\textit{р}}$ в робочій арматурі визначаємо з умови:
        \begin{equation}
            0,3f_{p0,1k} \leq \sigma_{\textit{р}} \leq 0,9f_{p0,1k}
        \end{equation}
        $$0,3 \cdot 765 \leq \sigma_{\textit{р}} \leq 0,9 \cdot 765$$
        $$229,5 \leq \sigma_{\textit{р}} \leq 688,5$$
        $$\sigma_{\textit{р}} = 600\;\textit{МПа}$$
    \item Виконуємо перевірку $\xi \leq \xi_R$
        \begin{equation}
            \xi_R = \dfrac{\epsilon_{cu3.cd}}{\epsilon_{cu3.cd} - \epsilon_{so}}
        \end{equation}
        Де $\epsilon_{so} = \dfrac{f_{pd} + 400 - 0,9\sigma_{p}}{E_p} \cdot 1000 = \dfrac{637,5 + 400 - 0,9 \cdot 600}{190000} \cdot 1000 = 2,61$

        $$\xi_R = \dfrac{3}{3 - 2,61} = 7,86$$
        $$0,41 \leq 7,86$$
        Умова виконується.
    \item Визначаємо положення нейтральної вісі:
        \begin{equation}
            M_f = b_{eff} \cdot h_f^\prime \cdot f_{cd} \cdot (d - 0,5 \cdot h_f)
        \end{equation}
        $$M_f = (1,31 \cdot 0,03 \cdot 17 \cdot 10^6 \cdot (0,26 - 0,5 \cdot 0,03))/1000 = 163,60845\;\textit{кНм}$$
        $$M_{max} \leq M_f$$
        $$417,9 > 163,60845$$
        Нейтральна вісь проходить у ребрі.
    \item У двотавровому перерізі згинальний момент сприймається двома прямокутниками (в поличці на ребрі):
        \begin{equation}
            M_{max} = M_1 + M_2 \Rightarrow M_1 = M_{max} - M_2
        \end{equation}
        \begin{multline*}M_2 = f_{cd} \cdot (b_{eff} - b) \cdot h_f \cdot (d - 0,5 \cdot h_f) = 17 \cdot 10^6 \cdot (1,31 - 0,18) \cdot 0,03 \times \\ \times (0,26 - 0,5 \cdot 0,03)/1000 = 141,19\;\textit{кНм}
        \end{multline*}
        $$M_1 = M_{max} - M_2 = 417,9 - 141,19 = 276,71\;\textit{кНм}$$
    \item Обчислюємо значення коефіцієнта
        \begin{equation}
            \alpha_{m1} = \dfrac{M_1}{f_{pd} \cdot b \cdot \zeta_1}
        \end{equation}
        $$\alpha_{m1} = \dfrac{276,71 \cdot 10^3}{637,5 \cdot 0,10 \cdot 0,836} = 0,0029$$
        $\xi = \dfrac{x}{d} \Rightarrow x = \xi \cdot d = 0,01 \cdot 260 = 2,6\;\textit{мм}$

        $\zeta = 0,996$
        $$\xi \leq \xi_R$$
        $$2,6 \leq 7,86$$
        Умова виконується.
    \item Визначаємо необхідну площу поздовжньої напруженої робочої арматури ребер плити:
        \begin{equation}
            A_{sp} = A_{sp1} + A_{sp2} = \dfrac{M_1}{f_{pd} \cdot d \cdot \zeta_1} + \dfrac{M_2}{f_{pd} \cdot (d - 0,5 \cdot h_f)}
        \end{equation}
        \begin{multline*}
        A_{sp} = \dfrac{276,75 \cdot 10^3}{637,5 \cdot 10^6 \cdot 0,26 \cdot 0,836} +\\+ \dfrac{141,19 \cdot 10^3}{637,5 \cdot 10^6 \cdot (0,26 - 0,5 \cdot 0,03)} = 0,002901\;\textit{м}^2    
        \end{multline*}
        
        Приймаємо $2\varnothing32A800 + 2\varnothing32A800$, $A_{sp}^{\textit{факт}} = 32,18\;\textit{см}^2$.
    \item Обчислюємо відсоток армування для прийнятої поздовжньої напруженої арматури:
        \begin{equation}
            \mu = \left(\dfrac{A_{sp}^{\textit{факт}}}{A_b}\right) \cdot 100\%
        \end{equation}
        $$\mu = \left(\dfrac{32,18}{5400}\right) \cdot 100\% = 0,59\%$$
        Відсоток армування для прийнятої поздовжньої напруженої арматури\\($0,3\% \leq \mu \leq 0,8\%$) входить до оптимальних значень.
\end{enumerate}
\subsection{Розрахунок міцності похилих перерізів поздовжніх ребер плити}
\subsection{Розрахунок полички плити на місцевий вигин}
\subsection{Розрахунок втрат попереднього напруження}
\subsection{Розрахунок плити на утворення тріщин нормальних до поздовжньої осі}
\subsection{Розрахунок тріщиностійкості плити в стадії виготовлення і транспортування}
\subsection{Розрахунок плити за деформаціями}

\newpage
\section{Проектування кроквяної ферми}



\end{document} % конец документа