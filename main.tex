%!TEX TS-program = xelatex

\documentclass[a4paper,14pt]{article}

\usepackage{header}


\author{}
\title{}
\date{\today}

\begin{document} % конец преамбулы, начало документа
\section{Компонування поперечної рами будівлі}
\subsection{Компонування поперечної рами промислової будівлі}
\begin{enumerate}
\item Визначаємо висоту підкранової балки: при кроці 6 м:

$h_{\textit{п,б}}=1000$ мм

\item Визначити висоту над кранової $H_{\textit{в}}$ і підкранової $H_{\textit{н}}$ частин колони, повну висоту $H_{\textit{1}}$, $H$.

Вантажопідйомність $Q= 20\;{\textit{т}}$.

Висота $A=2400\;{\textit{мм}}$.

$H_{\textit{н}}=8600\;{\textit{мм}}$.

h кранового рельса =$70\;{\textit{мм}}$.

$H_{\textit{в}}=h_{\textit{п.б.}}+A+1000$=$1000+2400+100=3500\;\textit{мм}$.

$H_1=H_{\textit{н}}+H_{\textit{в}}=8000+3500=12100\;\textit{мм}$.

$H=H_1+150=12250\;\textit{мм}$.

Висота ферми при прольоті 18\;\textit{м}\;:

$H_{\textit{ф}}=2450\;\textit{мм}$.

\item Прив'язка "а" розбивочной осі ряду колон:

-- нульова прив'язка.

\item Призначити висоту перетину над кранової частини колони $h_{\textit{верхнє}}$:

При нульовій прив'язці --- $380\;{\textit{мм}}$.

$h_{\textit{нижнє}}=(\frac{1}{10}\ldots\frac{1}{14})H_\textit{н}=860\ldots614\;\textit{мм}$.

$b_{\textit{нижнє}},\;b_{\textit{верхнє}}=(\frac{1}{20}\ldots\frac{1}{25})H_\textit{н}=430\ldots344\;\textit{мм}$.

Вид колони --- наскрізна.

Так як :$H_1<\textit{10,8}\;\textit{м}$; $h_\textit{нижнє}<\textit{900}\;\textit{м}$; $Q<\textit{30}\;\textit{т}$, проліт до 24 м, то приймаємо розміри колони:

$h_{\textit{гілки}}=200\;\textit{мм}$.

$h_{\textit{н}}=1000\;\textit{мм}$.

$b_{\textit{нижнє}},\;b_{\textit{верхнє}}=400\;\textit{мм}$.
%Сделать рисунок
\end{enumerate}
\newpage
\section{Статичний розрахунок поперечної рами}




1)Збір навантаження:

%Сделать таблицу

%Рис. 3.1.  Статична розрахункова схема рами промислових будівель:
%Рис. 3.2. Розрахункова схема кроквяної конструкції при визначенні опорної реакції RA
Розрахунковий проліт рами:

$l_0=L_{\textit{цеха}}-2с=17000-2\cdot 200=16600\;{\textit{мм}}$

Визначення опорної реакції $R^{\textit{Пост}}_A$:

\begin{equation}
    R^{\textit{Пост}}_A=0,5\cdot g^{\textit{покр}}\cdot l_0 + {\textit{1,1}}\cdot {\textit{0,5}}\cdot G^{\textit{стр}}_{\textit{П}}
\end{equation}

де : $G^{\textit{стр}}_{\textit{П}}$ - маса кроквяної конструкції

$g^{\textit{покр}}$ - навантаження на покритті

 
\begin{equation}
    g^{\textit{покр}}=g_{\textit{р}}\cdot S_1
\end{equation}

де : $g_{\textit{р}}$ - розрахункове постійне навантаження на 1 м$м^{\textit{2}}$ плити покриття

$S_1$-крок поперечних рам в будівлі

$g^{\textit{покр}}={\textit{3,52}}\cdot 6 = {\textit{21,12}}\;\textit{кН/м}$

$R^{\textit{Пост}}_A=0,5\cdot {\textit{21,12}}\cdot {\textit{16,6}}+ {\textit{1,1}}\cdot {\textit{0,5}}\cdot 60={\textit{208,296}}\;{\textit{кН}}$

Снігове навантаження

\begin{equation}
    p^{\textit{сн}}=S_{\textit{m}}\cdot S
\end{equation}
\begin{equation}
    S_m=\gamma_{fm}\cdot S_0 \cdot C
\end{equation}

де : $\gamma_{fm}$- коеф. надійності для середн. періоду повтрюваності снігового навантаження Т = 60 років 

$S_0$ - характеристичне значення снігового навантаження на 1 м$м^{\textit{2}}$ для заданого району будівництва

C = 1 при відсутності даних про режим експлуатації будівлі с плоскою конструкцією покрівлі і розміщенням його на висоті H < ${\textit{0,5}}$ км над рівнем моря.

$S_m=1,04\cdot 1400\cdot 1=1456\;\textit{Па}=1,456\;{\textit{кН/м}}^2$

$p^{\textit{сн}}=1,456\cdot 6=8,736\;\textit{кН/м}$

\begin{equation}
    R^{\textit{cн}}_A=0,5\cdot p^{\textit{сн}}\cdot l_0
\end{equation}

$R^{\textit{cн}}_A=0,5\cdot 8,736 \cdot 16,6 = 72,51\;\textit{кН/м}$

Кранове навантаження

Проліт крана $L_k$=$\textit{16,6}\;\textit{м}$

Ширина крана $B=6300\;\textit{мм}$

База крана $K=4400\;\textit{мм}$

$H=2400\;\textit{мм}$

$B_1=260\;\textit{мм}$

$P^n_{max}$-навантаження коліс на підкранові рейки-$195\;\textit{кН}$

Вага візка - $8,5\;\textit{т}$

G - Вага крана з візком -$28,5\;\textit{т}$

Тип кранової рейки - КР70

%вставить рисунок линий влияния


\begin{equation}
    D_{max}=\gamma_{fm}\cdot \psi \cdot P^n_{max} \cdot \sum y_i
\end{equation}

де: $\gamma_{fm}$ - см. п. 7.9 %[]

$\psi$ - см. п. 7.22 %[]

$\sum y_i$ - Рис..

\begin{equation}
    D_{min}=\gamma_{fm}\cdot \psi \cdot P^n_{min} \cdot \sum y_i
\end{equation}

\begin{equation}
P^n_{min} = \frac {Q+G}{n_0}-P^n_{max}
\end{equation}

де : $n_o$ - кількість коліс на одній стороні крана

$D_{max}=1,1\cdot 0,85 \cdot 195 \cdot 1,95$ = $355,534\;\textit{кН}$

$P^n_{min} = \frac {200+285}{2}-195$=$47,5\;\textit{кН}$

$D_{min}=1,1\cdot 0,85 \cdot 47,5 \cdot 1,95$ = $86,6\;\textit{кН}$

Навантаження на раму від поперечного гальмування

\begin{equation}
    T=\gamma_\textit{соч}\cdot \gamma_f \cdot T^\textit{кол}_{n} \cdot \sum y_i
\end{equation}

Горизонтальне поперечне гальмівне навантаження від одного колеса 
   для кранів з гнучким підвісом вантажу 

\begin{equation}
    T^\textit{кол}_{n}=\frac {0,05\cdot (Q+Q_t)}{n_0}
\end{equation}

$T^\textit{кол}_{n}=\frac {0,05\cdot (20+8,5)}{2}=\textit{0,7125}\;\textit{т}=\textit{7,2}\;\textit{кН}$

$T=0,85\cdot 1,2 \cdot 7,2 \cdot 1,95 = \textit{14,32}\;\textit{кН}$

Навантаження від стінових панелей:

\begin{equation}
    G_\textit{стпн}= S \cdot Н_\textit{н}\cdot g
\end{equation}

$G_\textit{стпн}= 6 \cdot 8,6\cdot 2,8 = 144,48\;\textit{кНм}$

\begin{equation}
    G_\textit{стпн.в.}= S \cdot Н_\textit{в}\cdot g
\end{equation}

$G_\textit{стпн.в.}= 6 \cdot 3,5\cdot 2,8= 58,8\;\textit{кНм}$

Вітрове навантаження

Граничне розрахункове значення вітрового навантаження:

\begin{equation}
    W_\textit{m}= \gamma_{fm} \cdot W_\textit{0}\cdot C
\end{equation}

де :  $\gamma_{fm}$ ---  коефіцієнт надійності, в залежності від терміну повторності максимального значення вітрового тиску в роках. На 100 років --- $\gamma_{fm}$ = 1,14

$W_\textit{0}$ --- характеристичне значення вітрового тиску, залежне від району будівництва. $W_\textit{0} - 0,47\;\textit{кНм}^2$

$h=5\;\textit{м} = W_{5}= 0,47\cdot 0,4 = 0,188\;\textit{кНм}^2$

$h=10\;\textit{м} = W_{10}= 0,47\cdot 0,6 = 0,282\;\textit{кНм}^2$

$h=20\;\textit{м} = W_{20}= 0,47\cdot 0,85 = 0,399\;\textit{кНм}^2$

Еквівалентне вітрове навантаження $W_e$

\begin{equation}
    W_\textit{e}= \frac{2M_3}{H^2}
\end{equation}

$M_3 = \frac{0,188\cdot 12,25^2}{2}+\frac {1}{2}\cdot (0,308-0,188)\cdot 7,25 \cdot (\frac {2}{3}\cdot 7,25 + 5)= 18,4\;\textit{кНм}^2$ 

$W_\textit{e}= \frac{2\cdot 18,4}{12,25^2}= 0,245\;\textit{кНм}^2$ 

Активний вітер 

\begin{equation}
    W_\textit{a}= W_\textit{e} \cdot B\cdot C_{aer}\cdot \gamma_{fm} 
\end{equation}

$W_\textit{a}= 0,245 \cdot 6\cdot 0,8\cdot 1,14 = 1,341\;\textit{кН/м.п.}$   

Пасивний вітер

$W_\textit{п}= 0,245 \cdot 6\cdot 0,6\cdot 1,14 = 1,01\;\textit{кН/м.п.}$ 

Зосереджена сила на рівні верха колон по середньому вітряному тиску між $0,308\;\textit{кНм}^2$ і $0,337\;\textit{кНм}^2$

$W = (\frac{0,308+0,337}{2})\cdot 6 \cdot 2,45 \cdot (0,8+0,6)\cdot 1,14 = 7,57\;\textit{кН}$ 

%Вставить рисунок разреза ветра

Статична розрахунок поперечної рами 

%<<<<<<< HEAD
\begin{enumerate}
    \item Момент інерції відносно осі Y:
    \begin{equation}
        I_z = \frac{b\cdot h^3_\textit{в}}{12}+\frac{bh-(H_\textit{н}-h_\textit{в})^2}{2} 
    \end{equation}
    
    $$I_z = \frac{40\cdot 20^3}{12}+\frac{40\cdot 20-(100-20)^2}{2}= $$
\end{enumerate}
%=======

%>>>>>>> c66bf3aeb27969b09cea77f323300f71c1fbe802




\begin{equation}
    A_f\geqslant\frac{1,05N_{n,max}}{R_0-\gamma_mH_1} 
\end{equation}

$$A_f\geqslant\frac{1,05\cdot25}{200-2\cdot6}=0,14\;{\textit{м}^2}$$

\begin{equation}
    A_f\geqslant\frac{1,05N_{n,max}}{R_0-\gamma_mH_1} 
\end{equation}



\newpage
\section{Проектування колони одноповерхової промислової будівлі}
\subsection{Розрахунок поздовжньої арматури колони}

\begin{enumerate}
    \item Обчислюємо ексцентриситет:
        \begin{equation}
            e_0 = \frac{M}{N}+e_a
        \end{equation}
        де: \begin{itemize}
                \item $e_a = \frac{1}{600}+8600 = 14,3\;\textit{мм}$
                \item $e_a = \frac{1}{30}+200 = 6,6\;\textit{мм}$
            \end{itemize}
        Обираємо $e_a = 14,3\;\textit{мм}$
        $$e_0 = \frac{265,99}{219,688}+0,014 = 0,384\;\textit{м}$$
    \item Наведений радіус інерції перерізу підкранової частини двогілкової колони:
        \begin{equation}
            i_{red}^2 = \frac{c^2}{4(\frac{1+3c^2}{\psi^2n^2h^2})}
        \end{equation}
        де: $\psi^2 = 1,5$

                $n = \frac{H_\textit{н}}{S} = \frac{8,6}{2} = 4,3\;\textit{м}$

                $S = (8 \ldots 10)h = 10 \cdot 0,2 = 2\;\textit{м}$
           
        $$i_{red}^2 = \frac{0,8^2}{4(\frac{1+3 \cdot 0,8^2}{1,5 \cdot 4,3^2 \cdot 0,2^2})} = 0,05859\;\textit{м}$$
    \item Приведена гнучкість підкранової частини колони:
        \begin{equation}
            \lambda_{red} = \frac{l_0}{i_{red}^2}
        \end{equation}
        де: $l_0 = 1,5H_\textit{н} = 1,5 \cdot 8,6 = 12,9\;\textit{м}$
        $$\lambda_{red} = \frac{12,9}{0,05859} = 220,17$$
        
        Гранична гнучкість:
        \begin{equation}
            \lambda \lim = \frac{20ABC}{\sqrt{n}}
        \end{equation}
        де: $n = \frac{N}{A_cf_{cd}} = \frac{719,658 \cdot 10^3}{2(0,4 \cdot 0,2) \cdot 17 \cdot 10^6} = 0,265$

            $A = \frac{1}{(1+0,2\phi_{ef})} = \frac{1}{(1+0,2 \cdot 2)} = 0,71$

            $\phi_{ef} = 2$

            $B = 1,1$

            $C = 0,7$
            $$\lambda\lim = \frac{20 \cdot 0,71 \cdot 1,1 \cdot 0,7}{\sqrt{0,265}} = 21,61$$

        Так як, $\lambda_{red} > \lambda\lim$ слід враховувати вплив прогину на величину ексцентриситету повздовжньої сили. В цьому випадку в розрахунку замість $e_0$ необхідно використовувати
        величину $(\eta \cdot l_0)$, де 
        \begin{equation}
            \eta = \frac{1}{1 - \frac{N}{N_{cr}}}
        \end{equation}
        \begin{equation}
            N_{cr} = \frac{6,4E_{cm}}{l_0^2}[\frac{I}{\phi_l}(\frac{0,11}{0,1 + \frac{\sigma_e}{\phi_p}} + 0,1) + \alpha I_s]
        \end{equation}
        $I = 2[\frac{bh^3}{12} + bh(\frac{c}{2})^2] = 2[\frac{0,4 \cdot 0,2^3}{12} + 0,4 \cdot 0,2 (\frac{0,8}{2})^2] = 0,02613\;\textit{м}^4$

        $\phi_l = 1 + \beta \frac{M_1}{M} = 1 + 1 \cdot \frac{26,653}{265,99} = 1,1 < (1 + \beta)$

        $I_S = 2 \rho bh(\frac{c}{2})^2 = 2 \cdot 0,02 \cdot 0,4 \cdot 0,2 \cdot (\frac{0,8}{2})^2 = 0,000512\;\textit{м}^4$

        $\sigma_e = \frac{l_0}{h_\textit{н}} = \frac{12,9}{1} = 12,9\;\textit{м}$

        $\phi_p = 1$

        $\alpha = \frac{E_S}{E_{ct}} = \frac{210\;\textit{Па}}{32,5\;\textit{Па}} = 6,46$
        $$N_{cr} = \frac{6,4 \cdot 32500 \cdot 10^6}{12,9^2}[\frac{0,02613}{1,1}(\frac{0,11}{0,1 + \frac{12,9}{1}} + 0,1) + 6,46 \cdot 0,000512]$$
        $$N_{cr} = 7354530\;\textit{Па} = 7354,53\;\textit{кН}/\textit{м}^2$$
        $$\eta = \frac{1}{1 - \frac{719,658}{7354,53}} = 1,11$$
    \item Визначаємо зусилля в гілках колони:
        \begin{equation}
            N_{\textit{в}1,2} = 0,5N \pm \frac{M \cdot \eta}{c}
        \end{equation}
        $$N_{\textit{в}1,2} = 0,5\cdot 719,658 + \frac{265,99 \cdot 1,1}{0,8} = 713,2\;\textit{кН}$$
        \begin{equation}
            M_\textit{в} = V \frac{S}{4}
        \end{equation}
        $$M_\textit{в} = 41,052 \cdot \frac{2}{4} = 20,526\;\textit{кН}$$
    \item Для кожної з гілок визначаємо:
        \begin{equation}
            e_0 = \frac{M_\textit{в}}{N_\textit{в}} + l_a
        \end{equation}
        \begin{equation}
            e = e_0\eta +  0,5h - a
        \end{equation}
        $\eta = 1$

        $h = 200\;\textit{мм}$

        $a = 30\;\textit{мм}$

        $d = h - a = 200 - 30 = 170\;\textit{мм}$

        $l_a = 200 / 30 = 6,6\;\textit{мм}$

        $\frac{S}{600} = \frac{2000}{600} = 3,33\;\textit{мм}$
        $$e_0 = \frac{20,526}{713,2} + 0,0066 = 0,035\;\textit{м}$$
        $$e = 0,035 \cdot 1 +  0,5 \cdot 0,2 - 0,03 = 0,105\;\textit{м}$$
    \item Підбираємо армування при несиметричному армуванні:
        \begin{equation}
            A_S^\prime = \frac{N \cdot e - 0,4 \cdot f_{cd} \cdot b \cdot d^2}{f_{yd} \cdot (d - a^\prime)} \geqslant 0
        \end{equation}
        $$A_S^\prime = \frac{713,2 \cdot 10^3 \cdot 0,105 - 0,4 \cdot 17 \cdot 10^6 \cdot 0,4 \cdot 0,17^2}{365 \cdot 10^6 \cdot (0,17 - 0,03)} \geqslant 0$$
        $$A_S^\prime = -0,0000728376\;\textit{м}^2$$

        Висновок --- переріз арматури приймаємо конструктивно.
        \begin{equation}
            A_S = \frac{0,55 \cdot f_{cd} \cdot b \cdot d - N}{f_{yd}} + A_S^\prime
        \end{equation}
        $$A_S = \frac{0,55 \cdot 17 \cdot 10^6 \cdot 0,4 \cdot 0,17 - 713,2 \cdot 10^3}{365 \cdot 10^6} +(-0,0000728376)$$
        $$A_S = -0,000284892\;\textit{м}^2$$

        Висновок --- переріз арматури приймаємо конструктивно.
\end{enumerate}

\subsection{Розрахунок розпірки двогілкової колони}

\subsection{Розрахунок колони із площини поперечної рами}
\newpage
\section{Проектування позацентрового навантаження фундаменту під колону}
\subsection{Визначення розмірів фундаменту і армування його плитної частини}
\subsection{Проектування підколонника фундаменту}

\newpage
\section{Проектування плити покриття}
\subsection{Розрахунок міцності поздовжніх ребер плити покриття за нормальними перерізами}
\subsection{Розрахунок міцності похилих перерізів поздовжніх ребер плити}
\subsection{Розрахунок полички плити на місцевий вигин}
\subsection{Розрахунок втрат попереднього напруження}
\subsection{Розрахунок плити на утворення тріщин нормальних до поздовжньої осі}
\subsection{Розрахунок тріщиностійкості плити в стадії виготовлення і транспортування}
\subsection{Розрахунок плити за деформаціями}

\newpage
\section{Проектування кроквяної ферми}



\end{document} % конец документа