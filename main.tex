%!TEX TS-program = xelatex

\documentclass[a4paper,14pt]{article}

\usepackage{header}


\author{}
\title{}
\date{\today}

\begin{document} % конец преамбулы, начало документа
\section{Компонування поперечної рами будівлі}
\subsection{Компонування поперечної рами промислової будівлі}
\begin{enumerate}
\item Визначаємо висоту підкранової балки: при кроці 6 м:

$h_{\textit{п,б}}=1000$ мм

\item Визначити висоту над кранової $H_{\textit{в}}$ і підкранової $H_{\textit{н}}$ частин колони, повну висоту $H_{\textit{1}}$, $H$.

Вантажопідйомність $Q= 20\;{\textit{т}}$.

Висота $A=2400\;{\textit{мм}}$.

$H_{\textit{н}}=8600\;{\textit{мм}}$.

h кранового рельса =$70\;{\textit{мм}}$.

$H_{\textit{в}}=h_{\textit{п.б.}}+A+1000$=$1000+2400+100=3500\;\textit{мм}$.

$H_1=H_{\textit{н}}+H_{\textit{в}}=8000+3500=12100\;\textit{мм}$.

$H=H_1+150=12250\;\textit{мм}$.

Висота ферми при прольоті 18\;\textit{м}\;:

$H_{\textit{ф}}=2450\;\textit{мм}$.

\item Прив'язка "а" розбивочной осі ряду колон:

-- нульова прив'язка.

\item Призначити висоту перетину над кранової частини колони $h_{\textit{верхнє}}$:

При нульовій прив'язці --- $380\;{\textit{мм}}$.

$h_{\textit{нижнє}}=(\frac{1}{10}\ldots\frac{1}{14})H_\textit{н}=860\ldots614\;\textit{мм}$.

$b_{\textit{нижнє}},\;b_{\textit{верхнє}}=(\frac{1}{20}\ldots\frac{1}{25})H_\textit{н}=430\ldots344\;\textit{мм}$.

Вид колони --- наскрізна.

Так як :$H_1<\textit{10,8}\;\textit{м}$; $h_\textit{нижнє}<\textit{900}\;\textit{м}$; $Q<\textit{30}\;\textit{т}$, проліт до 24 м, то приймаємо розміри колони:

$h_{\textit{гілки}}=200\;\textit{мм}$.

$h_{\textit{н}}=1000\;\textit{мм}$.

$b_{\textit{нижнє}},\;b_{\textit{верхнє}}=400\;\textit{мм}$.
%Сделать рисунок
\end{enumerate}
\newpage
\section{Статичний розрахунок поперечної рами}




1)Збір навантаження:

%Сделать таблицу

%Рис. 3.1.  Статична розрахункова схема рами промислових будівель:
%Рис. 3.2. Розрахункова схема кроквяної конструкції при визначенні опорної реакції RA
Розрахунковий проліт рами:

$l_0=L_{\textit{цеха}}-2с=17000-2\cdot 200=16600\;{\textit{мм}}$

Визначення опорної реакції $R^{\textit{Пост}}_A$:

\begin{equation}
    R^{\textit{Пост}}_A=0,5\cdot g^{\textit{покр}}\cdot l_0 + {\textit{1,1}}\cdot {\textit{0,5}}\cdot G^{\textit{стр}}_{\textit{П}}
\end{equation}

де : $G^{\textit{стр}}_{\textit{П}}$ - маса кроквяної конструкції

$g^{\textit{покр}}$ - навантаження на покритті

 
\begin{equation}
    g^{\textit{покр}}=g_{\textit{р}}\cdot S_1
\end{equation}

де : $g_{\textit{р}}$ - розрахункове постійне навантаження на 1 м$м^{\textit{2}}$ плити покриття

$S_1$-крок поперечних рам в будівлі

$g^{\textit{покр}}={\textit{3,52}}\cdot 6 = {\textit{21,12}}\;\textit{кН/м}$

$R^{\textit{Пост}}_A=0,5\cdot {\textit{21,12}}\cdot {\textit{16,6}}+ {\textit{1,1}}\cdot {\textit{0,5}}\cdot 60={\textit{208,296}}\;{\textit{кН}}$

Снігове навантаження

\begin{equation}
    p^{\textit{сн}}=S_{\textit{m}}\cdot S
\end{equation}
\begin{equation}
    S_m=\gamma_{fm}\cdot S_0 \cdot C
\end{equation}

де : $\gamma_{fm}$- коеф. надійності для середн. періоду повтрюваності снігового навантаження Т = 60 років 

$S_0$ - характеристичне значення снігового навантаження на 1 м$м^{\textit{2}}$ для заданого району будівництва

C = 1 при відсутності даних про режим експлуатації будівлі с плоскою конструкцією покрівлі і розміщенням його на висоті H < ${\textit{0,5}}$ км над рівнем моря.

$S_m=1,04\cdot 1400\cdot 1=1456\;\textit{Па}=1,456\;{\textit{кН/м}}^2$

$p^{\textit{сн}}=1,456\cdot 6=8,736\;\textit{кН/м}$

\begin{equation}
    R^{\textit{cн}}_A=0,5\cdot p^{\textit{сн}}\cdot l_0
\end{equation}

$R^{\textit{cн}}_A=0,5\cdot 8,736 \cdot 16,6 = 72,51\;\textit{кН/м}$

Кранове навантаження

Проліт крана $L_k$=$\textit{16,6}\;\textit{м}$

Ширина крана $B=6300\;\textit{мм}$

База крана $K=4400\;\textit{мм}$

$H=2400\;\textit{мм}$

$B_1=260\;\textit{мм}$

$P^n_{max}$-навантаження коліс на підкранові рейки-$195\;\textit{кН}$

Вага візка - $8,5\;\textit{т}$

G - Вага крана з візком -$28,5\;\textit{т}$

Тип кранової рейки - КР70

%вставить рисунок линий влияния


\begin{equation}
    D_{max}=\gamma_{fm}\cdot \psi \cdot P^n_{max} \cdot \sum y_i
\end{equation}

де: $\gamma_{fm}$ - см. п. 7.9 %[]

$\psi$ - см. п. 7.22 %[]

$\sum y_i$ - Рис..

\begin{equation}
    D_{min}=\gamma_{fm}\cdot \psi \cdot P^n_{min} \cdot \sum y_i
\end{equation}

\begin{equation}
P^n_{min} = \frac {Q+G}{n_0}-P^n_{max}
\end{equation}

де : $n_o$ - кількість коліс на одній стороні крана

$D_{max}=1,1\cdot 0,85 \cdot 195 \cdot 1,95$ = $355,534\;\textit{кН}$

$P^n_{min} = \frac {200+285}{2}-195$=$47,5\;\textit{кН}$

$D_{min}=1,1\cdot 0,85 \cdot 47,5 \cdot 1,95$ = $86,6\;\textit{кН}$

Навантаження на раму від поперечного гальмування

\begin{equation}
    T=\gamma_\textit{соч}\cdot \gamma_f \cdot T^\textit{кол}_{n} \cdot \sum y_i
\end{equation}

Горизонтальне поперечне гальмівне навантаження від одного колеса 
   для кранів з гнучким підвісом вантажу 

\begin{equation}
    T^\textit{кол}_{n}=\frac {0,05\cdot (Q+Q_t)}{n_0}
\end{equation}

$T^\textit{кол}_{n}=\frac {0,05\cdot (20+8,5)}{2}=\textit{0,7125}\;\textit{т}=\textit{7,2}\;\textit{кН}$

$T=0,85\cdot 1,2 \cdot 7,2 \cdot 1,95 = \textit{14,32}\;\textit{кН}$

Навантаження від стінових панелей:

\begin{equation}
    G_\textit{стпн}= S \cdot Н_\textit{н}\cdot g
\end{equation}

$G_\textit{стпн}= 6 \cdot 8,6\cdot 2,8 = 144,48\;\textit{кНм}$

\begin{equation}
    G_\textit{стпн.в.}= S \cdot Н_\textit{в}\cdot g
\end{equation}

$G_\textit{стпн.в.}= 6 \cdot 3,5\cdot 2,8= 58,8\;\textit{кНм}$

Вітрове навантаження

Граничне розрахункове значення вітрового навантаження:

\begin{equation}
    W_\textit{m}= \gamma_{fm} \cdot W_\textit{0}\cdot C
\end{equation}

де :  $\gamma_{fm}$ ---  коефіцієнт надійності, в залежності від терміну повторності максимального значення вітрового тиску в роках. На 100 років --- $\gamma_{fm}$ = 1,14

$W_\textit{0}$ --- характеристичне значення вітрового тиску, залежне від району будівництва. $W_\textit{0} - 0,47\;\textit{кНм}^2$

$h=5\;\textit{м} = W_{5}= 0,47\cdot 0,4 \cdot = 0,188\;\textit{кНм}^2$

$h=10\;\textit{м} = W_{10}= 0,47\cdot 0,6 \cdot = 0,282\;\textit{кНм}^2$

$h=20\;\textit{м} = W_{20}= 0,47\cdot 0,85 \cdot = 0,399\;\textit{кНм}^2$

Еквівалентне вітрове навантаження $W_e$

\begin{equation}
    W_\textit{e}= \frac{2M_3}{H^2}
\end{equation}

$M_3 = \frac{0,188\cdot 12,25^2}{2}+\frac {1}{2}\cdot (0,308-0,188)\cdot 7,25 \cdot (\frac {2}{3}\cdot 7,25 + 5)= 18,4\;\textit{кНм}^2$ 

$W_\textit{e}= \frac{2\cdot 18,4}{12,25^2}= 0,245\;\textit{кНм}^2$ 

Активний вітер 

\begin{equation}
    W_\textit{a}= W_\textit{e} \cdot B\cdot C_{aer}\cdot \gamma_{fm} 
\end{equation}

$W_\textit{a}= 0,245 \cdot 6\cdot 0,8\cdot 1,14 = 1,341\;\textit{кН/м.п.}$   

Пасивний вітер

$W_\textit{п}= 0,245 \cdot 6\cdot 0,6\cdot 1,14 = 1,01\;\textit{кН/м.п.}$ 

Зосереджена сила на рівні верха колон по середньому вітряному тиску між $0,308\;\textit{кНм}^2$ і $0,337\;\textit{кНм}^2$

$W = (\frac{0,308+0,337}{2})\cdot 6 \cdot 2,45 \cdot (0,8+0,6)\cdot 1,14 = 7,57\;\textit{кН}$ 

%Вставить рисунок разреза ветра

Статична розрахунок поперечної рами 

Кек чебурек




\begin{equation}
    A_f\geqslant\frac{1,05N_{n,max}}{R_0-\gamma_mH_1} 
\end{equation}

$$A_f\geqslant\frac{1,05\cdot25}{200-2\cdot6}=0,14\;{\textit{м}^2}$$

\begin{equation}
    A_f\geqslant\frac{1,05N_{n,max}}{R_0-\gamma_mH_1} 
\end{equation}
\newpage
\section{Проектування колони одноповерхової промислової будівлі}
\subsection{Розрахунок поздовжньої арматури колони}


\subsection{Розрахунок розпірки двогілкової колони}

\subsection{Розрахунок колони із площини поперечної рами}
\newpage
\section{Проектування позацентрового навантаження фундаменту під колону}
\subsection{Визначення розмірів фундаменту і армування його плитної частини}
\subsection{Проектування підколонника фундаменту}

\newpage
\section{Проектування плити покриття}
\subsection{Розрахунок міцності поздовжніх ребер плити покриття за нормальними перерізами}
\subsection{Розрахунок міцності похилих перерізів поздовжніх ребер плити}
\subsection{Розрахунок полички плити на місцевий вигин}
\subsection{Розрахунок втрат попереднього напруження}
\subsection{Розрахунок плити на утворення тріщин нормальних до поздовжньої осі}
\subsection{Розрахунок тріщиностійкості плити в стадії виготовлення і транспортування}
\subsection{Розрахунок плити за деформаціями}

\newpage
\section{Проектування кроквяної ферми}



\end{document} % конец документа